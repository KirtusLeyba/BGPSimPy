\section{Background}

The AS-layer is the highest level of organization of the Internet, with each individual AS represents a separately administered network, for example, a large ISP, a university network, or a government entity. The Border Gateway Protocol (BGP) is used to route traffic among ASes \cite{bgp}.  AS-level routes are known as \emph{paths} and are specified by routing tables stored locally in each AS. These tables express known connections to other ASes and local preferences for how data are routed, e.g., if there are multiple paths to a single destination, an AS may prefer the path that costs the least.

Because BGP is a distributed protocol, with all connectivity
information stored in the local routing tables, it is challenging to
infer the exact AS topology at any point in time.  Several strategies have been developed to infer AS topology, but none of them is perfect.  Methods that rely
on routing table information and BGP
updates suffer from the fact that BGP paths stored in routing tables
reflect only the current knowledge of a single AS about the routes
available~\cite{BGPStream}.
% FIXME: Not sure what BGPStream citation is for.  We need a different citation for RIPE unless that is what BGPStream is.
Thus one would need to use many sources to infer an accurate
picture of the BGP topology, which is not easily scalable for
global studies, particularly those looking at multiple time periods.
Similarly, methods based on empirical traceroute data are known to be incomplete or inaccurate~\cite{tracerouteProblems}.

\par AS-level studies are further complicated by the lack of ground
truth data for AS relationships and BGP paths. For AS relationships,
inference is often used based on economic considerations to classify
the relationships of ASes. This was first done by Gao \cite{gao} by
maximizing the occurances of certain economic rules on the AS graph by
choosing a particular set or relationships. This technique was only
evaluated against a single ISPs set of true relationships, however. As
part of the CAIDA project, this inference technique was extended,
leading to the CAIDA AS-relationship dataset \cite{CAIDApaper},
\cite{CAIDA}. We choose these relationships for our purposes, and they
are the current research standard.

\par Finding BGP paths is more of
a challenge. Packet based simulations, wherein BGP is simulated
directly, are computationally infeasible for the scales relevant to a
global study. An accurate, but realistic, simulation technique must be
chosen to provide useful research potential in this regard.  The
BGPSim algorithm \cite{quicksand} is one such simulation technique
that is suitable for this study's purposes. BGPSim takes a set of AS
relationships as input, such as those provided by the CAIDA dataset
\cite{CAIDA}, and returns routing trees based off of these
relationships. These paths are found via a modified breadth-first
seach (BFS) algorithm. The BFS adds edges to routing trees first
according to local preference (LP), then shortest path (SP), and
finally tiebreak (TB).  A resulting routing tree contains all the
equally reasonable paths (according to economic concerns) that exist
between source ASes (within the routing tree) to destination ASes (the
root of the routing tree). We find this technique suitable for our
purposes, so we extend BGPSim (in ways explained further in this
paper) as part of \toolname{}.
