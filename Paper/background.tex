\section{Background}

% We already said this in the introductino
%The AS-layer is the highest level of organization of the Internet, with each
%individual AS represents a separately administered network, for example, a
%large ISP, a university network, or a government entity.
The Border Gateway
Protocol (BGP) is used to route traffic among ASes \cite{bgp}.  AS-level routes
are known as \emph{paths} and are specified by routing tables stored locally in
each AS. These tables express known connections to other ASes and local
preferences for how data are routed, e.g., if there are multiple paths to a
single destination, an AS may prefer the path that costs the least.

Because BGP is a distributed protocol, with all connectivity
information stored in the local routing tables, it is challenging to
infer the exact AS topology at any point in time.  Several strategies
have been developed, each with advantages and drawbacks. For example,
methods based on empirical traceroute data are known to be incomplete
or inaccurate~\cite{tracerouteProblems}. Similarly, methods that rely
on routing table information reflect only the current knowledge of a
single AS, or small set of ASes, about available
routes~\cite{BGPStream}, and they rely on ASes that are willing to
share their routing table information.  Many such ASes are required
% FIXME: Not sure what BGPStream citation is for.  We need a different citation for RIPE unless that is what BGPStream is.
to infer a reliable picture of the BGP topology, an approach that does not scale well for global studies over time. 

\par AS-level studies are further complicated by the lack of ground
truth data for AS relationships and BGP paths. Two ASes might enter a peering relationship,
exchanging traffic but not announcing their connection. For AS relationships such as these, economic models are often used to classify
the relationships of ASes. This approach was pioneered by Gao \cite{gao}, who
assigned AS relationships that optimized certain economic rules on the AS graph.
%This technique was only
%evaluated against a single ISPs set of true relationships, however. As
This inference technique was extended by the CAIDA project,
leading to the CAIDA AS-relationship dataset \cite{CAIDApaper},
\cite{CAIDA}, which are the current research standard.
%We choose these relationships for our purposes, and they
%are the current research standard.

\par Knowing the AS topology and AS relationships is critical for finding BGP paths, but it is not sufficient because BGP does not always adopt
the shortest paths available on the
AS graph. Instead, they are determined by a combination of the dynamics of
BGP updates between ASes and the local preferences of each AS to route traffic
to destinations according to local AS policies.
%Packet based simulations, wherein BGP is simulated
%directly, are computationally infeasible for the scales relevant to a
%global study.
BGPSim is a realistic, simulation technique developed to infer likely BGP paths.~\cite{quicksand}. BGPSim takes as input a set of AS
relationships, such as those provided by the CAIDA dataset
\cite{CAIDA}, and returns routing trees based on these
relationships. A routing tree contains all
equally reasonable paths (according to economic and connectivity constraints) that between source ASes (within the routing tree) to destination ASes (the
root of the routing tree). These paths are found using a modified breadth-first
search (BFS) algorithm. The BFS adds edges to routing trees first
according to local preference (LP), then shortest path (SP), and
finally tiebreak (TB).
%We find this technique suitable \Ben[inline]{Why is it suitable?} for our
%purposes.
We adopt the basic algorithm of BGPSim
%(in ways explained further in this
%paper)
as part of \toolname{}.

The interplay between the Internet and policy has a history as long as the
Internet itself, but in terms of national-scale interference with packets the
first prominent case of a state-sponsored system to achieve this is the
so-called ``Great Firewall of China,'' which was first documented in the
technical research literature in 2006 by Clayton \emph{et
al.}~\cite{Clayton:2006:IGF:2166520.2166523}. Large-scale Internet surveillance is also prominent and has a
long history, but surveillance is not as easy to measure as censorship so only
information dumps such as the 2013 Snowden revelations can give us a sense of
the prevalence and time periods related to surveillance.  National-scale
manipulation of packets for reasons other than censorship, for example to inject
malware or carry out distributed denial-of-service attacks, was first seen being
carried out by China's ``Great Cannon'' in 2015~\cite{badtraffic}.  More
recently Egypt was documented doing similar injections using a commercial
product~\cite{191996}.  This phenomena is more targeted and therefore more
stealthy than national-scale censorship, so may be more prevalent than just
these two countries.
