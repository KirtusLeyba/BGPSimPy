\section{Border ASes and Chokepoint Potential}

In this section we motivate the choice to focus our analysis on border ASes and then
introduce and define out measure of chokepoint potential. We additionally introduce
the aggregate measure of chokepoint potential, national chokepoint potential (NCP)
for ranking and comparing countries according to the strength of their border AS chokepoints.

\begin{figure}
	\centering
	\includegraphics[width=\linewidth]{bnodes}
	\caption{The ratio of ASes that are border ASes plotted over several years.}\label{fig:bnodes}
\end{figure}

\begin{figure}
	\centering
	\includegraphics[width=\linewidth]{chokepoint}
	\caption{Chokepoint potential example. ASes A,B, and C are all border ASes. 
						AS D is an internal AS. ASes E and F are both external ASes.
						The out-to-in chokepoint potentials of A,B, and C are 0.25, 0.25, and 0.5 respectively.}\label{fig:chokepoint}
\end{figure}

\subsection{Border ASes}

Border ASes are ASes that lie adjacent from an AS registered to a different
country along a routing path. More formally, we define $g(\cdot)$ as the
mapping from ASes to the country of registration, and say an AS $a_u$ is a border
AS if $g(a_u) \neq g(a_v)$, where $a_v$ is a neighboring AS.  Because of their
position along national border, these ASes, can act as convenient locations for
national governments to control information flow into or out of a country. This
is especially true if a large number of BGP paths traverse a small number of
border ASes. Even as countries expand their Internet infrastructure, they may
be concentrating traffic that crosses national borders to a small number of
governmentally controlled ASes.  Evidence for this can be seen in the ratio of
the number of border ASes of a country to all ASes in that country.
%hints toward the abillity of that country to intercept information flows across
%their borders.
\figurename \ref{fig:bnodes} shows that globally the number of border ASes has
grown at a slower rate than the number of non-border ASes.
%over time so that the ratio of border ASes to ASes has almost continuously
%decreased. 
It would seem that this fact reflects an ever increasing capability for countries
to intercept AS-level paths.  However, an important consideration is not only
how many border nodes exist, but also how often each is utilized in BGP
routing. Because AS-level paths are generated by distributed BGP routing tables
and the local preferences of AS administrators, a better way to understand the
potential for bottlenecks on border ASes is to investigate the number of BGP
generated paths that pass through each AS. We propose an intuitive measure of
AS bottlenecks called \emph{chokepoint potential}. Surveillance and censorship
generally occurs at a national level so we further define an aggregate measure
for countries, which we call \emph{national chokepoint potential}.

% % Why Border ASes?
% In order to identify AS chokepoints and compare countries, we need a measure that can be calculated from the many paths between
% ASes. First, we decided to use a measure that is evaluated only on border ASes, or ASes that lie one hop from an AS belonging to another country. 
% The evolution of AS relationships and national boundaries on the AS graph hints that border ASes are an important feature in regards to the flow
% of information. Consider that while the number of ASes globally has continued to grow rapidly, the number of border ASes has grown more slowly.
% This result is depicted in figure \ref{fig:bnodes}. Additionally, while internal chokepoints may intercept many paths, those paths are required
% to have entered through a border AS (in the case of out-to-in paths) or exit through a border AS (for in-to-out paths). This focus on border ASes has
% the added benefit of making calculations more efficient.

\subsection{Chokepoint Potential}

We define a BGP path of length $n$ from source AS $a_0$ to destination AS $a_n$
as a sequence $p_{1\rightsquigarrow n} = \{a_i : i \in [1,n]\}$.  We define a
\textit{routing tree} to be a tree rooted at a single destination AS and
composed of every acceptable BGP path from the various source ASes that can
reach the destination AS. Note that if multiple paths exist from a source
to a destination, the routing tree is technicailly a DAG, but we use the traditional convention and call
all such data structures routing trees. The
chokepoint potential of a border AS is the ratio of the number of paths in (or out of) the country that contain the border AS to
the total number of paths in (or out of) the same country.
%intercepted by all border ASes belonging to
%the same country.
Formally, for country $c$, let $P = \{p_{i \rightsquigarrow
j} : g(a_i)=c \text{ and } g(a_j)\neq c\} $, i.e.\ the set of all paths
originating in $c$ and ending outside $c$. Then the chokepoint potential,
$cp(\cdot)$, of a border AS $a_u$ with $g(a_u) = c$ is
\begin{equation}
  \label{eq:chokepoint}
  cp(a_u) = \frac{|\{p: p \in P \text{ and } a_u \in P\}|}{|P|}
\end{equation}


The above definition applies to paths originating in country
$c$ and routing to a different country, which we will refer to as
\textit{outgoing} chokepoint potential or $cpo$. We similarily define
\textit{incoming} chokepoint potential or $cpi$, by redefining $P$ as $\{p_{i
\rightsquigarrow j}: g(a_i) \neq c \text{ and } g(a_j) = c\}$. That is all
paths originating outside of $c$ but terminating in $c$ and recomputing
Equation~\ref{eq:chokepoint}. 

%For instance, take
%a border AS $b$, belonging to country $c$.  If $P_c$ is the set of paths in to
%or out of country $c$ and $B_p$ is the set of border ASes within path $p$, then
%the chokepoint potential of $b$, $CP(b)$ is defined formally in equation
%\ref{eqn:chokePointPotential}. 
%This is calculated seperately for in-to-out paths (those starting from a source
%in the country in question) and out-to-in paths.
The incoming chokepoint
potential for a countries is illustrated in \figurename
\ref{fig:chokepoint}. We define both $cpo$ and $cpi$ because past work
has revealed that country-level paths are often asymmetric, meaning that the
forward path from AS $a$ to AS $b$ does not necessarily match the reverse path
\cite{characterizingAndAvoiding}.

% SF says 'paths' in response to Ben
Given a set of BGP paths, chokepoint potential is an intuitive way to
compare individual ASes. Note that the sum of the chokepoint potentials
for all border ASes for a given country is 1.0.
%, as in, all of the border ASes
%collectively control the flow of information over the country's border.
We define the aggregate \emph{national chokepoint potential} as the number of
border ASes required to control a particular fraction $f$ of paths. Formally,
given a country,$c$, with a set of $k$ border ASes $\{a_1,a_2,\cdots,a_k\}$ with
$cp(a_i)>cp(a_j) \forall i<j$ (i.e.\ sorted in descending order), we find the smallest $j$ such that 

\begin{equation}
  \label{eq:nationalchokepoint}
  \sum_{i=1}^j cp(a_i) > f
\end{equation}

and define national chokepoint potential as $CP(c,f) = 1/j$. The
multiplicative inverse preserves a clearer semantic meaning i.e.\ higher
chokepoint potential implies more control over paths. The fewer border ASes
required to control $f$ paths, the easier it is to perform
censorship or surveillance, e.g., by restricting ISPs,
placing filtering hardware, etc. 
%Thus, we can define another measure, with
%this one being a national measure. We define the \emph{national breakthrough
%potential} of a country to be the minimum number of ASes required to intercept a
%selected percentage of AS-level paths. 
Previous work used $f=0.90$ as a threshold indicating strong control of information
\cite{throats}, so we use this percentage as well. When used here, however,
90\% path control indicates the percentage of paths into (our out of) a country.
%that a single country intercepts 90\% of the paths
%that either enter said country (for the out-to-in case) or exit said country (for
%the in-to-out case).
National chokepoint potential is calculated
%by taking the
%strongest ASes first, such that the measure
as the smallest number of
border ASes needed to control $f$ of the paths.\footnote{We note that a dual measure
could be defined as the fraction of paths controlled by $j$ ASes. Our results
are substantively the same for either definition.}

We measure the absolute number of ASes required to control a specific
fraction rather than a percentage within a country to make
comparisons between countries more equitable.
Consider, for instance, the U.S., which has many more ASes than
most countries, and China, which has a much smaller number of ASes. If the United States and China 
were to both require the same
percentage of border ASes to intercept 90\% of paths, the similarity of these
nations would be misleading because the cost of controlling the larger absolute number of ASes
would be significantly higher. Our technique provides a comparison more likely to
reflect reality. 
%This measure is called a breakthrough potential because it
%indicates the number of border ASes that intercept a vast majority of paths,
%meaning that a country with high NBP would require controlling many border ASes
%to prevent a breakthrough, or an unfettered access of internal ASes to that
%country.
