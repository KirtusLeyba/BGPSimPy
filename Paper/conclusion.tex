\section{Conclusions and Discussion}

% Chokepoint Potential
In this paper we have introduced two novel measures, chokepoint potential, on single border ASes,
and national breakthrough potential, on entire nations, that together allow us to
evaluate the level of difficulty that a government faces when trying to censor the Internet
based entirely on AS-level topology. Our technique for evaluating AS-level topology using
\toolname{} provides a monitoring capability of BGP dynamics that was previously not possible.
We have taken advantage of efficient simulation, standard AS relationship datasets, and cross-platform
design principles so that this tool is readily deployable for future research.
% Ecolution over time
\par
Our application of \toolname{} to investigate evolution of the global AS graph over time shows us
interesting trends in AS topologies. We have identified the apparent lack of major nations to develop
more open Internet borders and gateways, despite the increasing connectivity and size of the AS graph. Additionally,
we find that the chokepoints in some nations known to conduct censorship and surveillance have increased in strength.
% Internet Freedom
\par
We have validated national breakthrough potential as having a relationship to actual censorship by showing
the significance of the relationship between the chokepoint potential of nations and the evaluations of freedom provided
by both freedom on the net and freedom of the press. This relationship implies that when censorship and surveillance are being evaluated
at the national level, it is worthwhile to consider the border AS chokepoints that are resident to the nation in question.

\subsection{Routing Trees Dataset}
In the hope to further AS topology research, we have open sourced the routing tree datasets generated in this study. While we generated routing trees
with an efficient algorithm, it still requires considerable time to calculate them, particulary for multiple timestamps. Additionally, each set of routing trees
takes up on the order of 50GB of disk space. By releasing these datasets we hope that researchers looking for a particular set of routing trees will find
working with these simpler than recalculating them. This also provides an alternative to research projects that might otherwise use measurement tools to estimate
BGP paths.

% Points of Presence
\subsection{Other Chokepoint Measures}
There are several ways that a country can create chokepoints, either intentionally or accidentally: taking advantage of Internet Exchange Points (IXPs),
limiting the number of physical connections that cross the border, centralizing
the DNS infrastructure within the country, or working
with---and supporting---a set of ISPs that have international connections.
Each of these combines two basic elements: limiting the number of organizations that the government must interact with and creating physical locations where traffic can be controlled.
We chose to define chokepoint potential in terms of the AS graph instead of physical locations because the AS graph captures the most interesting physical attributes of the Internet, and
virtualization of the physical and datalink layers on the Internet make physical locations less meaningful. A single undersea cable can be multiplexed to serve as several different links in
the AS graph, and ASes in one country can have physical Points of Presence (PoPs) in several other countries (e.g., China Telecom in Pasadena, California). Thus, the AS graph is a good approximation
of overall chokepoint potential, and information such as geography and the physical locations of routers does not add meaningfully to the analysis without considering every individual piece of Internet
infrastructue on a case-by-case basis. Additionally, the dramatic shifts in chokepoint potential over time depicted in our analysis are unrelated to constant geography, so our AS-level study is important
to show time evolution of chokepoints.

The DNS infrastructure is another potential graph where chokepoints can be
exploited for censorship and surveillance\~cite{Greschbach2016a} and is at a
higher layer in the network stack than IP routes.  However, we leave analysis of
this graph for future work because it is outside the scope of \toolname{} and
because the countries that rely on DNS as their main form of censorship are
fewer in number and are among the least sophisticated in terms of their
censorship regime.

\subsection{Future Work}
% OONI data
While linking chokepoint potential to FOTN scores is a substantial contribution, it still stands that FOTN can serve only as a proxy for censorship.
This is useful for large trends and general understanding, but a more fine-grained approach could be used to interpret the direct connection between
actual censorship events and the shifts in the AS graph. The Open Observatory of Network Interference, or OONI, \cite{OONI} provides Internet users around
the world with a tool called the ooniprobe. The ooniprobe lets users run a suite of tests to identify censorship anamolies of various types, and the results are
recorded in the large OONI database. Matching up changes in OONI measurements, such as increased censorship campaigning in an authoritarian nation, with shifts
in chokepoint potential would be a major step in understanding the interplay of censorship and AS-level chokepoints. This process involves designing a way to classify
censorship events and chokepoint potential changes, and as such lies beyond the scope of this study.
