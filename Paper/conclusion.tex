\section{Discussion}

\subsection{Routing Trees Dataset}
Despite the use of efficient simulation techniques, collecting BGP path data is still not a simple task.
For a global, multiple timestamp study such as this one, compute resources can be a limiting factor to
researchers without access to a compute cluster.
In the hope to further AS topology research, we have open sourced the routing tree datasets produced for
this study, available at [removed for peer review] generated in this study. By releasing these datasets we 
hope that researchers looking for a particular set of routing trees will find working with these simpler 
than recalculating them, or creating new datasets by extensive measurement or
inference.

% Points of Presence
\subsection{Other Chokepoint Measures}
There are several ways that a country can create chokepoints, either intentionally or accidentally: taking advantage of Internet Exchange Points (IXPs),
limiting the number of physical connections that cross the border, centralizing
the DNS infrastructure within the country, or working
with---and supporting---a set of ISPs that have international connections.
Each of these combines two basic elements: limiting the number of organizations that the government must interact with and creating physical locations where traffic can be controlled.
We chose to define chokepoint potential in terms of the AS graph instead of physical locations because the AS graph captures the most interesting physical attributes of the Internet, and
virtualization of the physical and datalink layers on the Internet make physical locations less meaningful. A single undersea cable can be multiplexed to serve as several different links in
the AS graph, and ASes in one country can have physical Points of Presence (PoPs) in several other countries (e.g., China Telecom in Pasadena, California). Thus, the AS graph is a good approximation
of overall chokepoint potential, and information such as geography and the physical locations of routers does not add meaningfully to the analysis without considering every individual piece of Internet
infrastructure on a case-by-case basis. Additionally, the dramatic shifts in chokepoint potential over time depicted in our analysis are unrelated to constant geography, so our AS-level study is important
to show time evolution of chokepoints.

The DNS infrastructure is another potential graph where chokepoints can be
exploited for censorship and surveillance~\cite{Greschbach2016a} and is at a
higher layer in the network stack than IP routes.  However, we leave analysis of
this graph for future work because it is outside the scope of \toolname{} and
because the countries that rely on DNS as their main form of censorship are
fewer in number and are among the least sophisticated in terms of their
censorship regime.

\subsection{Future Work}
% OONI data
While linking chokepoint potential to FOTN scores is a substantial contribution, it still stands that FOTN can serve only as a proxy for censorship.
This is useful for large trends and general understanding, but a more fine-grained approach could be used to interpret the direct connection between
actual censorship events and the shifts in the AS graph. The Open Observatory of Network Interference, or OONI, \cite{OONI} provides Internet users around
the world with a tool called the ooniprobe. The ooniprobe lets users run a suite of tests to identify censorship anomalies of various types, and the results are
recorded in the large OONI database. Matching up changes in OONI measurements, such as increased censorship campaigning in an authoritarian nation, with shifts
in chokepoint potential would be a major step in understanding the interplay of censorship and AS-level chokepoints. This process involves designing a way to classify
censorship events and chokepoint potential changes, and as such lies beyond the scope of this study.

\section{Conclusion}
% Chokepoint Potential
This paper addresses the question of how Internet structure relates to 
international boundaries and how it is changing over time.  By defining chokepoint
potential, both for individual border ASes and aggregated by country,
we provide a quantitative way to measure these effects in terms of the ease
with which a government can control routing paths across its borders. 
Our technique for generating BGP paths and evaluating chokepoints
using \toolname{} provides public domain software and routing trees for the entire AS graph.
%a monitoring capability of BGP dynamics that was previously not possible.
We have taken advantage of efficient simulation, standard AS relationship datasets, and cross-platform
design principles so that this tool will be readily deployable for future research.
% Ecolution over time
\par We used \toolname{} to investigate how the global AS graph has
evolved over time, finding interesting trends and outliers. These
studies reveal a tendency in large nations to concentrate paths across
their borders into a small number of ASes, despite the increasing
connectivity and size of the AS graph. Additionally, we find that border
chokepoints in some nations known to conduct censorship and
surveillance have steadily increased in strength.
% Internet Freedom
\par
We studied the relationship between NCP and two evaluations of openness, Freedom of the Net  and Freedom of the press, finding statistically significant relationships in both cases.
%These results suggest that it is worthwhile to consider border AS chokepoints when studying
This relationship implies that when censorship and surveillance are
being evaluated at the national level, it is worthwhile to consider
the border AS chokepoints that are resident to the nation in question.
