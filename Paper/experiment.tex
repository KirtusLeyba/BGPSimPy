\section{Experimental Setup}
\label{sec:exp}

% Chokepoint Evolution
%There is ongoing research interest into ASes that intercept large portions of
%Internet paths \cite{throats}, \cite{decoy}. Some important questions remain
%unanswered however. For instance, to what extent does the current state of the
%Internet support national governments' attempts to censor Internet traffic? How
%does this vary from nation to nation? Is the Internet developing more powerful
%chokepoints, or becoming more evenly accessible?

This section describes how we used \toolname{} to study three questions related to national borders and the BGP network:
(1) To
what extent do BGP networks today reflect national boundaries
%support
%national governments' ability to censor and surveil Internet traffic and
how do they vary across nations? (2) How has this structure evolved
over time? and (3) How do topological chokepoints correlate with
qualitative measures of Internet and press freedom?

All of our experiments used the relevant CAIDA AS relationship dataset for 
each timestamp we studied.  These contain inferred relationships
between ASes, which the simulation framework uses to
generate routing trees and  calculate chokepoint potentials.
We use the relationships to identify border ASes, and
we use the Team Cymru ASN lookup service to map each AS to the country
with which it is registered. \toolname{} returns both the set of
routing trees built via simulation and the chokepoint potential that
was determined for each AS. These data form the basis for
multiple levels of analysis: comparing chokepoint potential between countries,
assessing how many ASes are required to control most international BGP paths, 
and comparing results across multiple time points.
%timestamps to monitor the evolution of a nation's national chokepoint
%potential, indicating trends toward stronger chokepoints or,
%alteratively, more open borders.

% Chokepoints and Internet Freedom
Chokepoints provide one leverage point for nations interested in censorship or surveillance.
%potential can be leveraged to determine if a nation can
%intercept many BGP paths with few ASes, it is possible that there is some correlation between
Thus, we are interested in the extent to which the Internet freedom of a country correlates with its chokepoint potential.
%If a higher national chokepoint potential
%for a nation is related to a less free Internet or press, this indicates that topology is linked
%to censorship in some way, perhaps indicating that the flow of information through border ASes
%is an important factor in interpreting a nations ability to censor their traffic.
Although correlation certainly does not imply a causal relation, correlations can highlight overall trends and even provide clues about how censorship might be implemented in countries that are known to control content.
%direction cannot be inferred by such a relationship, it would strengthen the importance of chokepoint
%potential as a measure of the evolutionary direction of the Internet.
We tested the statistical relationship between national chokepoint potential
and two relevant qualitative evaluations of freedom. First, we used
Freedom House's Freedom On The Net report \cite{FOTN}. FOTN scores
quantify the level of Internet freedom in countries.  Each country
receives a numerical score from 0 (the most free) to 100 (the least
free). We compare our results to FOTN scores (for years 2014-2017) by ranking nations
according to the number of border ASes required to intercept 90\% of
paths for each nation.  FOTN includes only 65 nations in the most
recent report (2017), so we also compare to the Freedom of
the Press evaluation (2014-2017), which assesses 201 nations and territories.

% To probe these questions we first used our chokepoint evaluation technique to
% investigate the national chokepoint potential of all nations for the current Internet as
% well as the change in evolution of AS chokepoint potentials over time. We looked at multiple
% snapshots of AS relationships from the CAIDA dataset (2009-2018).  For each
% timestamp, we generated routing trees based on these relationships. Then we
% calculated the chokepoint potential for every border AS per snapshot using
% \toolname{}. As a result we can investigate how countries have changed over time
% in their capability to enact censorship and surveillance. This is an attempt to
% understand what topological trends have developed historically.  With this test
% we can compare nations, and see which ones have, over time, increased their
% capability to control the flow of information across their borders.

%%%%%%
