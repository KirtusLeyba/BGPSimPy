\section{Experimental Setup}

% Chokepoint Evolution
%There is ongoing research interest into ASes that intercept large portions of
%Internet paths \cite{throats}, \cite{decoy}. Some important questions remain
%unanswered however. For instance, to what extent does the current state of the
%Internet support national governments' attempts to censor Internet traffic? How
%does this vary from nation to nation? Is the Internet developing more powerful
%chokepoints, or becoming more evenly accessible?

In this section we use chokepoint potential to examine three questions: 1) to
what extent does the current state of the AS-level Internet topology support
national governments' ability to censor and surveil Internet traffic? 2) How
different is this ability vary across nations? and 3)  How has this ability
evolved over time?

To probe these questions we first used our chokepoint evaluation technique to
investigate the chokepoint potential of all nations for the current Internet as
well as the change in chokepoint potentials over time. We looked at multiple
snapshots of AS relationships from the CAIDA dataset (2012-2018).  For each
timestamp, we generated routing trees based on these relationships. Then we
calculated the chokepoint potential for every border AS per snapshot using
toolname{}. As a result we can investigate how countries have changed overtime
in their capability to enact censorship and surveillance. This is an attempt to
understand what topological trends have developed historically.  With this test
we can compare nations, and see which ones have overtime increased their
capability to control the flow of information across their borders.

% Chokepoints and Internet Freedom
If chokepoint potential can be leveraged to determine if a nation can easily
implement censorship, it stands to reason that their might already be a
negative relationship between the chokepoint potential of a nation and its
Internet freedom. If a significant relationship were to be found it would
strengthen chokepoint potential as a measure of a nation's censorship
capability and it would increase the value in monitoring chokepoint potentials
across the globe.

We tested whether a significant relationship exists between our measure of
national chokepoints and a qualitative evaluation of Internet Freedom. For
Internet Freedom we used the Freedom House's Freedom On The Net report
\cite{FOTN}. FOTN scores quantify the level of Internet freedom in countries.
Each country receives a numerical score from 0 (the most free) to 100 (the
least free). We compare our results to the FOTN scores by ranking nations
according to national breakthrough potential, particulary the number of border
ASes required to intercept 90\% of paths for each nation and viewing these
rankings against the FOTN scores.
