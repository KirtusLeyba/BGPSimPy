\section{Experimental Setup}
\label{sec:exp}

% Chokepoint Evolution
%There is ongoing research interest into ASes that intercept large portions of
%Internet paths \cite{throats}, \cite{decoy}. Some important questions remain
%unanswered however. For instance, to what extent does the current state of the
%Internet support national governments' attempts to censor Internet traffic? How
%does this vary from nation to nation? Is the Internet developing more powerful
%chokepoints, or becoming more evenly accessible?

In this section we use chokepoint potential to examine three questions: 1) To
what extent does the current state of the AS-level Internet topology support
national governments' ability to censor and surveil Internet traffic and how does
this ability vary across nations? 2) How has this ability
evolved over time? and 3) How does our topological chokepoint measure correlate to
qualitative measures of Internet and press freedom?

Our experiments begin with the CAIDA AS relationship dataset for a particular timestamp.
These detail inferred relationships between ASes, with which we can use our simulation framework
to generate routing trees and then calculates the chokepoint potentials for each AS. We use these 
relationships to identify border ASes, and we use the Team Cymru ASN lookup service to map each AS to the 
country with which it is registered. \toolname{} returns both the set of routing trees built via simulation
and the chokepoint potential that was determined for each AS. The resulting data allows us to conduct
multiple levels of analysis. The resulting chokepoint potentials allow us to compare nations in regards
to what kind of border topology they have. We can answer whether many or few ASes are needed for a 
particular nation to intercept a majority of BGP paths, for instance. We can also compare the results
of experiments at different timestamps to monitor the evolution of a nation's national chokepoint potential,
indicating trends toward stronger chokepoints or, alteratively, more open borders.

% Chokepoints and Internet Freedom
If chokepoint potential can be leveraged to determine if a nation can
intercept many BGP paths with few ASes, it is possible that there is some correlation between
this ability and the Internet freedom of that nation. If a higher national chokepoint potential
for a nation is related to a less free Internet or press, this indicates that topology is linked
to censorship in some way, perhaps indicating that the flow of information through border ASes
is an important factor in interpreting a nations ability to censor their traffic. While a causal
direction cannot be inferred by such a relationship, it would strengthen the importance of chokepoint
potential as a measure of the evolutionary direction of the Internet.

We tested whether a significant relationship exists between our measure of
national chokepoints and two qualitative evaluations of communications Freedom. For
Internet Freedom we used the Freedom House's Freedom On The Net report
\cite{FOTN}. FOTN scores quantify the level of Internet freedom in countries.
Each country receives a numerical score from 0 (the most free) to 100 (the
least free). We compare our results to the FOTN scores by ranking nations
according to national breakthrough potential, particulary the number of border
ASes required to intercept 90\% of paths for each nation and viewing these
rankings against the FOTN scores. FOTN only includes 65 nations in the most recent
report (2017), so we also compare our measure with Freedom of the Press, which assessed
201 nations and territories.

% To probe these questions we first used our chokepoint evaluation technique to
% investigate the national chokepoint potential of all nations for the current Internet as
% well as the change in evolution of AS chokepoint potentials over time. We looked at multiple
% snapshots of AS relationships from the CAIDA dataset (2009-2018).  For each
% timestamp, we generated routing trees based on these relationships. Then we
% calculated the chokepoint potential for every border AS per snapshot using
% \toolname{}. As a result we can investigate how countries have changed over time
% in their capability to enact censorship and surveillance. This is an attempt to
% understand what topological trends have developed historically.  With this test
% we can compare nations, and see which ones have, over time, increased their
% capability to control the flow of information across their borders.

%%%%%%
