\section{Introduction}

\par The Internet is rapidly changing from a collection of networks that span geo-political boundaries and promotes the free flow of information into a network that reflects political and economic constraints.
%Internet censorship and surveillance pose a
%significant threat to individuals throughout the world.
As routine
tasks, communications, entertainment, and information move online and
are mediated by the Internet, most of us have little choice about
whether or not to rely on the Internet.  According to the
International Telecommunication Union (ITU), the number of individual
Internet users increased from 1.024 billion in 2005 to 3.578 billion
in 2017 \cite{itu}. The majority of these users operate in an
environment that restricts Internet freedom in some way. For example,
the 2017 Freedom on the Net report from Freedom House reports that
64\% of Internet users belong to a nation with Internet that is not
free or partly free \cite{FOTN}.  Beyond censorship and surveillance,
the EU is considering content rules surrounding copyright, net
neutrality rules in the U.S.\ were recently overturned, and large
content services are under enormous pressure to control fake news and bad actors. Other forms of Internet
tampering such as code injection~\cite{badtraffic,191996} have been observed within the
Internet's backbone. Taken together, these
trends will likely restructure the Internet in unforeseen ways,
as organizations respond to new challenges and realities, particularly those imposed by legal and regulatory regimes.

Today, most such control is exercised at the country level, as
governments have recognized both the threat and the opportunity that
is posed by ubiquitous online communication. For example, much of the
organization, revolutionary momentum, and broadcasting of events
during the Arab Spring have been attributed to social media
communications through the Internet \cite{arabspring}, and several
countries sought to control these movements by disconnecting
in-country networks from the rest of the world \cite{BGPdisrupt}.
These two factors, increasing dependence on the Internet and increased
content monitoring, point to the need for improved tools and methods
for studying worldwide Internet structure over time. Previous research has
investigated country level AS topology \cite{DBLP:journals/corr/abs-0903-3218},
focusing on how the global AS graph layout indicates that interdomain routing
for many Internet users depends on the infrastructure of other nations. We take
a different approach by developing a technique to monitor the amount of paths intercepted
by border ASes, allowing us to compare nations according to how BGP paths enter and exit
each nation.
% Current
% methods have primarily focused on the node level (individual ASes)
% rather than the country level, 
% they have not described how AS topology
% has changed over time with respect to national boundaries, and
% existing BGP simulation methods make it difficult to address 
% questions such as these efficienty and at scale.

\par
In this paper, we focus on the Autonomous Systems (AS) level of the Internet.
Autonomous Systems are groups of routers under common management, such as the network of
a university or an Internet Service Provider (ISP).
%The change in national AS topologies 
%is of interest to censorship research.
The total size of the AS graph (The global network 
of all ASes) has grown from about 10,000 ASes in the early 2000s to over 60,000 today. 
The structural properties of national subgraphs have evolved differently
from nation to nation whether from economic decision making, infrastructural necessities, 
or efforts to build a powerful censorship and surveillance network \cite{economicPeering, irancensor}.
%The Autonomous Systems (AS) layer of the Internet has grown and changed dramatically over its history.
%The locations and relationships of these ASes determine how many chokepoints
%of AS-level routing exist and what strength these chokepoints have in regards to paths intercepted.
This rapid expansion, together with the changing role of national
governments, points to the importance of understanding properties like
path robustness, AS hierarchies, and the potential for organizations to
control information as it flows in and out of their networks.

\par Every nation has a different layout and number of ASes, and these
national networks connect to foreign ASes in unique ways. Considering
just censorship and surveillance, different nations use different
strategies for this form of content monitoring. For instance, China
both conducts keyword filtering in border ASes and in internal
provincial ASes \cite{chinafiltering}, while Iran routes its Internet
traffic through a centralized facility \cite{irancensor}.  A picture
of a nation's content monitoring capabilities then is not complete
without examining its underlying AS topology.  Having the ability to
investigate the AS topology of a nation will both help researchers
identify nations that could, for example, easily conduct censorship
and also provide possible insight into what kind of censorship is
likely being conducted.  More generally, we are interested in the extent to which the Internet is changing from a borderless structure into one that reflects national boundaries. 

%We examine how censorship and surveillance might be conducted 
%at the AS-level by
We focus on border ASes, or ASes that can connect directly to at least
one AS from another nation and introduce a measure called \emph{chokepoint potential}, which quantifies the percentage of BGP paths into or out of a country which pass through that AS.  At the country level, we are interested in aggregating chokepoint potential in different ways to quantify the tendency of all BGP paths crossing the border to pass through a small (or large) number of chokepoints (border ASes). We sometimes refer to this as \emph{national chokepoint potential}.
%We can then aggregate this measure and define \emph{national breakthrough potential} to
%measures the concentration of routing paths into and out of a country. 
%\Ben[inline]{These definitions are important. They need to be a concise and clear definitions. I don't think it's right yet, but we are close}
With these measures 
we can ask how the AS-level topology has changed over time with respect
to national boundaries. 

We developed a
suite of tools, called \toolname{}, for studying national chokepoints on
the AS graph efficiently. To illustrate these ideas and tools, we
study how national chokepoint potential correlates with two independent
measures of civil liberty, finding that a significant relationship exists between national chokepoint potential and each measure.
The paper extends earlier research on AS topology in several ways: it introduces chokepoint potential, it presents open-source cross-platform tools and datasetes for simulating Border Gateway Protocol (BGP) paths,
determining national chokepoint potential, and analyzing changes over time.  In addition, we report and analyze data for several countries of interest.
%Our measure of border ASes, chokepoint potential, defined in detail
%in section 3, is a succinct way to estimate the important properties of
%border ASes related to the ability of a nation to censor and surveil, and our aggregate extension of this measure to
%the national level, national breakthrough potential, is an effective way to evaluate and
%compare nations and analyze the evolution of the AS-level censorship and surveillance landscape.\Ben[inline]{This paragraph seems a little redundant now. It is a combo of other previous writing, but I am not sure what to cut.}

%Whether or not AS-level topology supports
%these efforts is a research question pivotal to an understanding of
%the dynamics of Internet censorship and surveillance.

%% \par
%% The Internet has been used as a tool for the citizens of authoritarian nations to voice opinions,
%% organize revolutionary movements, and connect with other nations' governments and citizens
%% to seek aid. Much of the organization, revolutionary momentum, and broadcasting
%% of happenings during the Arab Spring can be attributed to social media communications
%% through the Internet \cite{arabspring}. Because of this potential, national governments may
%% take interest in maximizing their ability to control the flow of information on the Internet.
%Censorship and Surveillance are growing phenomena

%The Internet is Dynamic and growing



%Current Events

%Diverse Censorship Architecture


% Major Contributions
\par
Specifically, the main contributions of the paper are: 
\begin{enumerate}
 \item The chokepoint potential measure for single border ASes and the 
 aggregate measure of national chokepoint potential, motivated as a way to measure the extent to which AS topology reflects national boundaries.

 \item A study of how national AS-level chokepoints have changed over the past nine years, demonstrating that the Internet has evolved to facilitate stronger
 AS-level chokepoints for some countries and lead to more open flow of information across borders for others.
 \item An open-source tool, \toolname{}, and datasets for simulating BGP paths and evaluating chokepoints efficiently at different time points for the entire Internet.  
 \item A study that reveals a significant relationship between chokepointpotential and Internet freedom, as measured by two qualitative sources.  This result suggests that chokepoint potential can be interpreted as an indicator of a country's capability for conducting censorship or surveillance of its international Internet traffic.
\end{enumerate}


% Rest of Paper Layout
\par
The layout of the rest of this paper is as follows: Section 2 provides relavent background information
to the problems we are investigating; Section 3 introduces and defines the measures of chokepoint
potential and national breakthrough potential; Section 4 details our tool for simulating and evaluating 
AS topologies \toolname{}; Section 5
explains our experimental setups; Sections 6, 7, and 8 provide results, discussion, and related work
respectively.
