\section{Introduction}

\par Widespread Internet censorship and surveillance pose a
significant threat to individuals throughout the world. As routine
tasks, communications, entertainment, and information move online and
are mediated by the Internet, most of us have little choice about
whether or not to rely on the Internet.  According to the
International Telecommunication Union (ITU), the number of individual
Internet users increased from 1.024 billion in 2005 to 3.578 billion
in 2017 \cite{itu}. The majority of these users operate in an
environment that restricts Internet freedom in some way. For example,
the 2017 Freedom on the Net report from Freedom House reports that
64\% of Internet users belong to a nation with Internet that is not
free or partly free \cite{FOTN}.  Beyond censorship and surveillance,
the EU is considering content rules surrounding copyright, net
neutrality rules in the U.S.\ were recently overturned, and large
content services are under enormous pressure to 'do
something' about fake news and bad actors. Also, other forms of Internet
tampering such as code injection~\cite{badtraffic,191996} have been observed within the
Internet's backbone. Taken together, these
trends will likely restructure the Internet in unforeseen ways,
as organizations respond to new challenges and realities.

Today, most such control is exercised at the country level,
as governments have recognized both the threat and the opportunity
that is posed by ubiquitous online communication. For example, much
of the organization, revolutionary momentum, and broadcasting of
events during the Arab Spring have been attributed to social media
communications through the Internet \cite{arabspring}. Several countries
sought to control these movements by disconnecting in-country networks from the rest of the world \cite{BGPdisrupt}.
These two factors, increasing levels of content monitoring and
Internet dependence, point to the need for improved tools and methods
for studying worldwide Internet censorship and surveillance over time.
\todo[inline]{A SENTENC OR TWO HERE SUMMARIZING THE STATE OF THE FIELD AND WHAT'S
  MISSING.}  

\par
In this paper, we focus on the Autonomous Systems (AS) level of the Internet.
Autonomous Systems are groups of routers under common management, such as the network of
a University or an Internet Service Provider (ISP). We apply AS-level topologies to
censorship and surveillance by evaluating border ASes, or ASes that can connect directly to at least
one AS from another nation according to a new measure \emph{chokepoint potential}.
With this measure we can ask how AS topology has changed over time with respect
to national boundaries.  We define \emph{national breakthrough potential} to
measure how a country's AS topology is organized in terms of paths
that can carry traffic across international borders. We developed a
suite of tools, called \toolname{}, for studying national chokepoints on
the AS graph efficiently. To illustrate these ideas and tools, we
study how national chokepoint potential correlates with two independent
measures of civilian liberty, finding that a significan relationship exists between NBP and both liberty measures.
The paper extends earlier research on AS topology, first by introducing the NBP measure, 
then by presenting open-source cross-platform tools for simulating Border Gateway Protocol (BGP) paths,
determining breakthrough potential at the country level, and analyzing how this measure has changed over time for certain countries of interest.


%Whether or not AS-level topology supports
%these efforts is a research question pivotal to an understanding of
%the dynamics of Internet censorship and surveillance.

%% \par
%% The Internet has been used as a tool for the citizens of authoritarian nations to voice opinions,
%% organize revolutionary movements, and connect with other nations' governments and citizens
%% to seek aid. Much of the organization, revolutionary momentum, and broadcasting
%% of happenings during the Arab Spring can be attributed to social media communications
%% through the Internet \cite{arabspring}. Because of this potential, national governments may
%% take interest in maximizing their ability to control the flow of information on the Internet.
%Censorship and Surveillance are growing phenomena

%The Internet is Dynamic and growing

The change in national AS topologies is of interest to censorship research.
The total size of the AS graph (The global network of all ASes) has grown from about 10,000 ASes in the early 2000s to over 60,000 today. 
The structural properties of national subgraphs have evolved differently
from nation to nation whether from economic decision making, infrastructural necessities, or efforts to build a powerful censorship and surveillance
network.
%The Autonomous Systems (AS) layer of the Internet has grown and changed dramatically over its history.
%The locations and relationships of these ASes determine how many chokepoints
%of AS-level routing exist and what strength these chokepoints have in regards to paths intercepted.
This rapid expansion, together with the changing role of national
governments, points to the importance of understanding properties like
path robustness, AS hierarchies, and the potential for organizations to
control information as it flows in and out of their networks.

%Current Events

%Diverse Censorship Architecture
\par Every nation has a different layout and number of ASes, and these national networks
connect to foreign ASes in unique ways. The
censorship and surveillance strategies of nations also differ. For
instance, China both conducts keyword filtering in border ASes and in
internal provincial ASes \cite{chinafiltering}, while Iran routes its
Internet traffic through a centralized facility \cite{irancensor}.
A picture of a nation's censorship or surveillance capabillities then
is not complete without a topological view point.
Having the capabillity to investigate the AS topology of a nation, then, will both help
researchers identify nations that could easily conduct censorship and
also provide possible insight into what kind of censorship is likely
being conducted. Our measure of border ASes, chokepoint potential, defined in detial
in section 3, is a succint way to estimate the important properties of
border ASes related to these capabilities, and our aggregate extension of this measure to
the national level, national breakthrough potential, is an effective way to evaluate and
compare nations and analyze the evolution of the AS-level censorship and surveillance landscape.

% Major Contributions
\par
The primary contributions of our work are as follows: 
\begin{enumerate}
 \item We introduce the measure of chokepoint potential for single border ASes and the aggregate measure national breakthrough potential, 
motivating their value as a way to interpret the capability for a nation to enact censorship or surveillance of its Internet traffic.
 \item We provide an overview of the evolution of national AS-level chokepoints over time, showing how the evolution of the Internet has either facillitated stronger
 AS-level chokepoints or lead to more open flow of information across borders.
 \item We develop a new tool, \toolname{}, that provides the capability for
efficiently evaluating chokepoints for a given state of the Internet. 
 \item We show that national breakthrough potential
has a significant relationship to Internet freedom as measured by two qualitative sources.
\end{enumerate}


% Rest of Paper Layout
\par
The layout of the rest of this paper is as follows: Section 2 provides relavent background information
to the problems we are investigating; Section 3 introduces and defines the measures of chokepoint
potential and national breakthrough potential; Section 4 details our tool for simulating and evaluating AS topologies \toolname{}; Section 5
explains our experimental setups; Sections 6, 7, and 8 provide results, discussion, and related work
respectively.
