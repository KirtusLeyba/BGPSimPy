\section{Related Work}   
\par
Previous studies also used BGP path models to find ASes that
intercept a high fraction of paths, e.g.,~\cite{throats} which
reported that 90\% of paths on the Internet could be intercepted by only 30
or so ASes. This work generated paths starting with
paths to top websites
as defined by the Alexa top websites project,
and then appending additional edges from the full
AS graph according to the Gao algorithm \cite{gao}.
%The
%ASes that intercept many paths are often
%within nations that conduct censorship.
This work was extended
in \cite{decoy}, which showed how ASes that intercept many paths could
be used for decoy routing. Our paper extends this work in several ways.
%This approach for identifying AS level
%chokepoints has several potential pitfalls that are remedied by the approach
%taken in this paper.
First, 
%the authors in \cite{throats} did not consider that
many of the websites in the Alexa dataset are in China, due to its large
Internet population, potentially biasing results.
%which artificially inflates the effect of
%Chinese ASes.
while we consider paths between each
source-destination AS pair.  More importantly, our work studies how chokepoints have changed over time, rather than considering 
a single snapshot.
%This makes it difficult to
%discuss the evolution of Internet topology, and whether or not chokepoints are
%anomalous or common.
Finally, we compare chokepoints quantitatively between different
countries by defining a measure, rather than simply identifying those
countries with ASes that intercept the most paths. 

%the aforementioned approach for identifying
%% chokepoints does not allow a clear comparison between different nations. It
%% can be said that one nation controls a large portion of Internet paths, but
%% not how easily traffic directed through that country could be intercepted on a
%% national level. For this, some aggregate measure across all of
%the border ASes
%within a country must be considered, as it is in this paper.     
\par
In \cite{chinafiltering}, Xu et al. investigated the AS level topology of China
to identify where keyword filtering occurred. They found that the most
effective ASes
%with which to deploy keyword filtering devices
are those in the backbone of the Chinese AS topology. A relevant
contribution of \cite{chinafiltering} is that, while most filtering
occurs in border ASes, some filtering occurs in provincial ASes. China
had a diverse strategy for Internet censorship at the time, targeting
both chokepoints and the Chinese provincial network, but this may have
changed since. The potential for various forms of censorship in
regards to various AS level topologies motivates the question: Is
centralized censorship or decentralized censorship more common?
Instead of directly identifying censorship devices on the AS graph, we
instead have quantified the chokepoint potential of ASes on borders,
and then compared that to  Internet freedom measures.

\par We are not the first to investigate
the relationship between Internet freedom and AS-level
topology. Similar techniques have been used to classify nations
according to the connectivity of their ASes
\cite{politicsrouting} for a single snapshot in time,
limiting the results. Our work focuses on country-level
chokepoints as the link between Internet topology and censorship or
surveillance. Finally, our study includes
the U.S. and Russia which the earlier work did not.

Routing and its interplay with Internet censorship, surveillance, and related
issues is a general research area with a broad set of research questions.
Karlin \emph{et al.}~\cite{DBLP:journals/corr/abs-0903-3218} considered the
centrality of countries with respect to routing of other nations' traffic, a
related problem that is distinct from chokepoints to monitor/control traffic
into and out of a government's own country.  Dainotti \emph{et
al.}~\cite{Dainotti:2011:ACI:2068816.2068818} analyze two specific large-scale
Internet disruptions at the routing level.  Khattak \emph{et
al.}~\cite{Khattak:2014:LCI:2663716.2663750} performed a detailed analysis of
censorship at the ISP level in Pakistan.

While there have been attempts to characterize methods for
censoring Internet content~\cite{Tschantz2016a,Khattak2016a} there is no
comprehensive list of all the different ways a state actor can manipulate
traffic.  For some methods, the AS graph is relevant, such as IP address
blacklisting, traffic throttling~\cite{DBLP:journals/corr/Anderson13}, URL or
packet filtering, packet injection, physically shutting down infrastructure, and
BGP attacks.  For other methods, the AS graph is less relevant, such as creating
internal national networks~\cite{DBLP:journals/corr/abs-1209-6398}, portal
censorship such as search engine filtering and social media post deletion,
propaganda campaigns, and manipulation of the DNS system.  Our work is
complementary to other studies that focus more on Internet routing's effect on
high-level content than on a country's desire to enforce policies on all traffic
that crosses their borders, such as Edmundson \emph{et
al.}~\cite{Edmundson:2018:NHI:3209811.3211887}.

