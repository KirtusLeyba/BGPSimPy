\section{Related Work}   
\par
Previous studies have used BGP path models to find ASes that
intercept a high fraction of paths. One such project in \cite{throats}
identified that 90\% of paths on the Internet could be intercepted with only 30
or so ASes. The researchers in \cite{throats} generated paths by first using
only paths to top websites as defined by the Alexa top websites project,
and then appending additional edges to those paths from the full
AS graph according to the principles defined by Gao \cite{gao}. Many of the
ASes that were found to intercept a large number of paths were found to be
within nations that conduct censorship. As an extension of these results,
another paper \cite{decoy} revealed that ASes that intercept many paths could
be utilized for decoy routing. This approach for identifying AS level
chokepoints has several potential pitfalls that are remedied by the approach
taken in this paper. First, the authors in \cite{throats} did not consider that
many of the paths found from the Alexa top websites would have destinations in
nations that censor Internet traffic, particularly China due to its large
Internet population. This artificially inflates the chokepoint nature of
Chinese ASes. As an alternative, in this paper we consider paths from every
source-destination AS pair, and we make a distinction between in-to-out paths
and out-to-in paths. Secondly, chokepoints have previously only been
identified at a single snapshot of the Internet. This makes it difficult to
discuss the evolution of Internet topology, and whether or not chokepoints are
anomalous or common. Finally, the aforementioned approach for identifying
chokepoints does not allow a clear comparison between different nations. It
can be said that one nation controls a large portion of Internet paths, but
not how easily traffic directed through that country could be intercepted on a
national level. For this, some aggregate measure across all of the border ASes
within a country must be considered, as it is in this paper.     
\par
In \cite{chinafiltering}, Xu et al. investigated the AS level topology of China
to identify where keyword filtering occurred. They found that the most
effective ASes with which to deploy keyword filtering devices are those in the
backbone of the Chinese AS topology. A relevant contribution of
\cite{chinafiltering} is that, while most filtering occurs in border ASes,
some filtering at the time of the study was controlled by non-central provincial
ASes. China had a
diverse strategy for Internet censorship, both targeting chokepoints and the
Chinese provincial network, but this may have changed since. The potential for various forms of censorship in
regards to various AS level topologies motivates the question: Is centralized
censorship or decentralized censorship more common? Instead of directly
identifying censorship devices on the AS graph, we instead quantify the
chokepoint potential of ASes on the national level, and then compare that with
qualitative Internet freedom measures and empirical censorship events.
\par
We are not the first to investigate the relationship between Internet freedom and
AS-level topology. Similar techniques have been used to classify nations according
to the connectivity of their ASes \cite{politicsrouting}. This has only been done for a
single moment in time, however, making the results limited in terms of stability and predictability.
Additionaly, previous work has not used country level chokepoints as the link between Internet topology
and censorship or surveillance practices. The work in \cite{politicsrouting} chose to relate Internet topology
to the Freedom of the Press measure from freedom house instead of the Freedom On The Net score. They chose to do this
to include more nations. Additionally, this study did not include the United States and Russia in their experiments because they were
outliers in regards to their topologies. Through our approach we hope to extend this previous work by finding an interesting measure for
understanding the dynamics of all nations, as well as targeting our results more specifically to Internet freedom by using the Freedom On
The Net score as our measure for Internet freedom. Through our techniques, we provide a simple measure that not only 
sheds light on the relationship between topology and Internet freedom, but reveals currently free
nations that could easily implement censorship if their governments decided to.

