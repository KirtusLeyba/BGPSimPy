
\section{\toolname}

In this section we describe \toolname{}, its relationship to previous attempts
at identifying routing paths, and how it is used to calculate chokepoint
potential.

% Preprocessing and Data pipeline
\toolname{} is a new tool that simluates plausible BGP paths and calculates AS
chokepoint potential and national chokepoint potential.  \toolname{} uses
readily available AS relationship data, e.g. data made available from
CAIDA~\cite{CAIDA}, and uses a similar approach to BGPSim \cite{quicksand} to
generate a set of routing trees. \Ben[inline]{Need a short description of how
BGPSim works, maybe below.}. These routing trees, as well as a set of AS
country codes (for identifying which nation an AS belongs to) are used to
determine the chokepoint potentials border ASes.  Finally, \toolname{}
calclates the national breakthrough potential for each nation. In our
experiments, we used the country codes returned from Team Cymru's IP to ASN
whois service \cite{cymru} to determine which ASes were registered to which
nation.  This service only maintains the most recent registration of each AS,
so to control for uncertain country to AS links we keep ASes that have been
registered more recently than the current timestamp being tested in the path
simulation but we do not assign these ASes chokepoint
potentials. We chose this
route because the datasets of AS registrations for the various ASN registries
tended to have more missing data at specific timestamps.  Fortunately, ASes
with registration changes after each test date tend to not be in the AS graph
for that test. \Ben[inline]{Two things. 1) This is sort of confusing, see if you
can clarify it. 2) Is this a big deal? How often does this happen? If it
happens frequently do we expect it to affect our results? }

% Routing Tree Algorithm
\par In order to calculate the routing trees, we use an extended version of the
BGPSim algorithm developed by Gil et. al in \cite{quicksand}. In our work we
addressed the following limitations of BGPSim: (1) BGPSim returns a set of ASes
for each path it considers but not the order in which they are visited; (2)
Once routing trees are determined, they cannot be accessed later without
recalculation; (3) BGPSim relies on the outdated parallelization framework
DryadLinq for C\#. To address these issues, we implemented an our routing
simulation tool in Python. \Ben[inline]{Why does BGPSim not return ordered
paths, this seems trivial, but I don't know the implementation details, was it
just an oversight or did you do something cool?}\toolname{} returns ordered paths from its routing trees, saves
routing trees to disk after calculation, and is parallelized with MPI via the
mpi4py library. These improvements have the added benefit of yielding a cross
platform routing tree algorithm that is ready to work on most hardware.

% Chokepoint calculation
\par Once \toolname{} generates the routing trees, they can be processed to
determine chokepoint potentials. This is done by iterating over every path
between each AS-pair. Because we use the same random tie-break as BGPSim, there
is only 1 path between each AS-pair considered, even if multiple exist in the
routing tree.\Ben[inline]{Put stuff in about tiebreaking being rare and
unlikely to affect results} Once a path is determined, it is traversed. For each AS visited,
the number of paths intercepted by that AS is incremented. This is done for
both outgoing paths and incoming paths, so only one traversal is necessary
per path. Additionally, the number of paths of each type that belong to each
nation is tallied, as this makes up the denominator in equation
\ref{eqn:chokePointPotential}.  \toolname{} takes the resulting chokepoint
potentials and generates national breakthrough potentials for each nation.
