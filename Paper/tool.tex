
\section{\toolname}

% Preprocessing and Data pipeline
\toolname{} is a new tool that calculates the chokepoint potential for every border AS given a set of AS relationships as well as the national breakthrough potential
for all nations represented by the AS graph.
\toolname{} first takes the entire AS graph, as in from the CAIDA dataset \cite{CAIDA}, and uses the principles of BGPSim \cite{quicksand}
to generate a set of routing trees. These routing trees, as well as a set of AS country codes (for identifying which nation an AS belongs to)
are used to determine the chokepoint potentials for every border AS. Finally, \toolname{} calclates the national breakthrough potential 
for each nation. In our experiments, we used the country codes returned from Team Cymru's
IP to ASN whois service \cite{cymru} to determine which ASes were registered to which nation. 
This service only maintains the most recent registration of each AS, so to control for uncertain country to AS links
we keep ASes that have been registered more recently than the current timestam being tested in the path simulation but we do not assign these ASes chokepoint potentials. We 
chose this route because the datasets of AS registrations for the various ASN registries tended to have more missing data at specific timestamps.
Fortunately, ASes with registration changes after each test date tend to not be in the AS graph for that test.

% Routing Tree Algorithm
\par
In order to calculate the routing trees, we use an extended version of the BGPSim algorithm developed by Gil et. al in \cite{quicksand}. In
our work we addressed the following limitations of BGPSim: (1) BGPSim returns a set of ASes for each path it considers but not the order in which
they are visited; (2) Once routing trees are determined, they cannot be accessed later without recalculation; (3) BGPSim relies on the outdated parallelization
framework DryadLinq for C\#. To address these issues, we use a Python implementation we dub BGPSimPy. BGPSimPy returns ordered paths from its routing trees,
saves routing trees to disk after calculation, and is parallelized with MPI via the mpi4py library. These improvements have the added benefit of yielding a cross platform
routing tree algorithm that is ready to work on most hardware.

% Chokepoint calculation
\par
Once BGPSimPy generates the routing trees, they can be processed to determine chokepoint potentials. This is done by iterating over every path between each AS-pair. Because
we use the same random tie-break as BGPSim, there is only 1 path between each AS-pair considered, even if multiple exist in the routing tree. Once a path is determined, it is
traversed. For each AS visited, the number of paths intercepted by that AS is incremented. This is done for both in-to-out paths and out-to-in paths, so only one traversal is
necessary per path. Additionally, the number of paths of each type that belong to each nation is tallied, as this makes up the denominator in equation \ref{eqn:chokePointPotential}.
\toolname{} takes the resulting chokepoint potentials and generates national breakthrough potentials for each nation.