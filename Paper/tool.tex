
\section{\toolname}

This section describes \toolname{}, how it relates to earlier work, and how it is used to calculate chokepoint
potential.

% Preprocessing and Data pipeline
\toolname{} determines a set of plausible BGP paths through simulation and calculates AS
chokepoint potential and national chokepoint potential.  \toolname{} takes as input AS relationship data, such as that provided by
CAIDA~\cite{CAIDA}.  It also takes as input a set of AS
country codes (that identifies which nation an AS belongs to) for
determining chokepoint potential.  \toolname{}
uses an algorithm based on BGPSim \cite{quicksand} to
generate a set of routing trees. \toolname{} 
calculates both AS and national chokepoint potential for any country.

Our experiments used country codes returned from Team Cymru's IP to
ASN whois service \cite{cymru} to determine which ASes were registered
to which nation.  This service maintains only the most recent
registration of each AS.  To control for this possible source of
error, we retain ASes in the path simulation, even if they were
registered more recently than the current timestamp being studied. This is because
the relationships of these ASes and paths through these ASes are still valid.
We do not assign these ASes chokepoint potentials, as it is not clear which nation
these ASes are registered for. We took this approach
because the datasets of AS registrations for the various ASN
registries tend to have more missing data at specific timestamps.
We do not expect this to have a considerable effect on our results
because few ASes do not have a definite nation (in the best case in the most recent
test this is less than 1\% of ASes, in the worst case in the earliest test this is around 12\% of ASes.)
Additionally, not all of the ASes that are unlinked to a nation are border ASes, and those will not
affect our results at all.

% Routing Tree Algorithm
\par To calculate the routing trees, we use an extended version of the
BGPSim algorithm developed by Gil et. al in \cite{quicksand}. This
component of \toolname{} addresses several limitations that prevented
us using BGPSim directly: (1) BGPSim was not used to test statistics on paths, so it
doesn't return ordered paths for calculating AS-level statistics; (2)
Once routing trees are determined, they cannot be accessed later
without recalculation; (3) BGPSim relies on the outdated
parallelization framework DryadLinq for C\#. To address these issues,
we implemented our routing simulation in
Python. \toolname{} returns
ordered paths from its routing trees, saves routing trees to disk
after calculation, and is parallelized with MPI via the mpi4py
library. These improvements have the added benefit of yielding a cross
platform routing tree algorithm that is portable to most
hardware. The source code for \toolname{} and all datasets are publicly available at [URL blinded for peer review].


% Chokepoint calculation
\par Once \toolname{} generates the routing trees, they can be processed to
determine chokepoint potentials or by other researchers interested in other questions. To calculate chokepoint potential, \toolname{} iterates over every path
between each AS-pair. Because we use the same random tie-break method as BGPSim,this process returns 
exactly one path between each AS-pair considered, even if multiple options exist in the
routing tree. Tiebreaks do not have a noticeable effect on our results. For instance, the maximum
standard deviation for the chokepoint potential of any AS in 5 runs of one of our timestamps was 
$\leq$ 0.005, meaning the chokepoint potentials for each AS barely changed depending on the tiebreak
decisions.
Once the path has been determined, it is traversed to identify border nodes and increment their counts, both
for outgoing and incoming paths.  Thus, only one traversal is conducted
per path. Additionally, the number of paths of each type that belong to each
nation is tallied. \toolname{} takes the resulting chokepoint
potentials and generates national chokepoint potentials for each nation.

Calculating routing trees for the entire AS graph takes time proportional to the $|V|^2$ where $|V|$ is the total number of ASes; calculating chokepoint potential takes $|V|^2D$ where $D$ is the average depth of a routing tree. For example,
it took about 12 wall clock hours to compute all of the routing trees for the Jan. 2017 dataset, running on 4 Intel Xeon E5-2680 V4 CPUs.  For each dataset, this is a one-time cost because the routing trees are stored externally.  [URL blinded for peer review] 
