
\section{\toolname}

This section describes \toolname{}, how it relates to earlier work, and how it is used to calculate chokepoint
potential.

% Preprocessing and Data pipeline
\toolname{} determines a set of plausible BGP paths through simulation and calculates AS
chokepoint potential and national chokepoint potential.  \toolname{} takes as input AS relationship data, such as that provided by
CAIDA~\cite{CAIDA}.  It also takes as input a set of AS
country codes (that identifies which nation an AS belongs to) for
determining chokepoint potential.  \toolname{}
uses an algorithm based on BGPSim \cite{quicksand} to
generate a set of routing trees. \Ben[inline]{Need a short description of how
BGPSim works, maybe below.}. \toolname{} 
calclates both AS and national chokepoint potential for any country.

Our experiments used country codes returned from Team Cymru's IP to
ASN whois service \cite{cymru} to determine which ASes were registered
to which nation.  This service maintains only the most recent
registration of each AS.  To control for this possible source of
error, we retain ASes in thepath simulation, even if they were
registered more recently than the current timestamp being studied, but
we do not assign these ASes chokepoint potentials. We took this approach
because the datasets of AS registrations for the various ASN
registries tend to have more missing data at specific timestamps.
Fortunately, ASes with registration changes after each test date tend
to not be in the AS graph for that test. \Ben[inline]{Two things. 1)
  This is sort of confusing, see if you can clarify it. 2) Is this a
  big deal? How often does this happen? If it happens frequently do we
  expect it to affect our results? }

% Routing Tree Algorithm
\par To calculate the routing trees, we use an extended version of the
BGPSim algorithm developed by Gil et. al in \cite{quicksand}. This
component of \toolname{} addresses several limitations that prevented
us using BGPSim directly: (1) BGPSim returns a set of ASes for each
path it considers but not the order in which they are visited; (2)
Once routing trees are determined, they cannot be accessed later
without recalculation; (3) BGPSim relies on the outdated
parallelization framework DryadLinq for C\#. To address these issues,
we implemented our routing simulation in
Python. \Ben[inline]{Why does BGPSim not return ordered paths, this
  seems trivial, but I don't know the implementation details, was it
  just an oversight or did you do something cool?}\toolname{} returns
ordered paths from its routing trees, saves routing trees to disk
after calculation, and is parallelized with MPI via the mpi4py
library. These improvements have the added benefit of yielding a cross
platform routing tree algorithm that is portable to most
hardware.

\Steph[inline]{Where is the code available?  What about the dataset of routing trees?  that is one of our big contributions.}

% Chokepoint calculation
\par Once \toolname{} generates the routing trees, they can be processed to
determine chokepoint potentials or by other researchers interested in other questions. To calculate chokepoint potential, \toolname{} iterates over every path
between each AS-pair. Because we use the same random tie-break method as BGPSim,this process returns 
exactly one path between each AS-pair considered, even if multiple options exist in the
routing tree.\Ben[inline]{Put stuff in about tiebreaking being rare and
unlikely to affect results} Once the path has been determined, it is traversed to identify border nodes and increment their counts, both
for outgoing and incoming paths.  Thus, only one traversal is conducted
per path. Additionally, the number of paths of each type that belong to each
nation is tallied, which becomes the denominator in equation
\ref{eqn:chokePointPotential}.  \toolname{} takes the resulting chokepoint
potentials and generates national chokepoint potentials for each nation.

\Steph[inline]{A brief complexity analysis, even just in words, would help, as woulddata on how long the computations take.  Complexity discussion here, timing data in Experiments?}
